\documentclass[../dissertation.tex]{subfiles}

\begin{document}
\section{Scene}
\begin{itemize}
    \item Designing Emergency Checklists is difficult
    \item Procedures in checklists must be tested in simulators~\cite{nasa-design},
      which usually means trained pilots test it, as the tests need
      to work consistently~\cite{manifesto} (making sure it's not lengthy,
      concise and gets critical procedures)
    \item Checklists are usually carried out in high
        workload environments, especially emergency ones
\end{itemize}

\section{Motivation}
\begin{itemize}
  \item Testing procedures in checklists are often neglected~\cite{nasa-design}
    \item There are some checklists that may not be fit
        for certain scenarios - e.g. ditching (water landing)
        checklist for US Airways Flight 1549 assumed at least one engine
        was running~\cite{AWE1549}, but in this scenario, there were none
    \item Some checklists may make pilots \enquote{stuck}
      - not widely implemented, could be fixed with \enquote{opt out} points.
      e.g. US Airways 1549, plane below 3000ft, could have skip to
      later in the checklist to something like turn on APU, otherwise plane
      will have limited control~\cite{AWE1549}.
    \item Checklists may take too long to carry out - Swissair 111
\end{itemize}

\section{Aim}
The goal of this project is to test checklists in Quick Reference Handbooks (QRH)
for flaws that could compromise the aircraft and making sure that the tests can
be completed in a reasonable amount of time by pilots. It is also crucial to make
sure that the tests are reproducible in the same flight conditions and a variety of
flight conditions.

\section{Objectives}
\begin{enumerate}
  \item Research current checklists that may be problematic and are testable
    in the QRH tester being made
  \item Implement a formal model that runs through checklists, with the
    research gathered to produce an accurate test
    \begin{enumerate}
      \item Understand the relative states of the aircraft that need to be captured
      \item Ensure that the results of the checklist procedures are consistent
    \end{enumerate}
  \item Implement a QRH tester manager that
    \begin{itemize}
      \item Runs the formal model and reacts to the output of the formal model
      \item Connect to a flight simulator to run actions from the formal model
      \item Implement checklist procedures to be tested, run them, and get
        feedback on how well the procedure ran
    \end{itemize}
\end{enumerate}

\end{document}

