\documentclass[../dissertation.tex]{subfiles}

\begin{document}
\section{Scene}
% \begin{itemize}
%     \item Designing Emergency Checklists is difficult
%     \item Procedures in checklists must be tested in simulators~\cite{nasa:design},
%       which usually means trained pilots test it, as the tests need
%       to work consistently~\cite{manifesto} (making sure it's not lengthy,
%       concise and gets critical procedures)
%     \item Checklists are usually carried out in high
%         workload environments, especially emergency ones
% \end{itemize}

Designing aviation checklists is difficult and requires time to test them
in simulators and the real world.~\cite{nasa:design}
The simulators require trained pilots to test the checklist and make sure that they
work consistently~\cite{manifesto}; testing that the steps in the checklist
are concise, achieves the goal of the checklist, and will not take too
long to complete to the point it could compromise the safety of the aircraft.
These checklists are also carried out by the crew in high workload environments,
where this workload would be elevated if an emergency were to occur.~\cite{caa:emergency}

\section{Motivation}
% \begin{itemize}
%   \item Testing procedures in checklists are often neglected~\cite{nasa:design}
%     \item There are some checklists that may not be fit
%         for certain scenarios - e.g. ditching (water landing)
%         checklist for US Airways Flight 1549 assumed at least one engine
%         was running~\cite{ntsb:AWE1549}, but in this scenario, there were none
%     \item Some checklists may make pilots \enquote{stuck}
%       - not widely implemented, could be fixed with \enquote{opt out} points.
%       e.g. US Airways 1549, plane below 3000ft, could have skip to
%       later in the checklist to something like turn on APU, otherwise plane
%       will have limited control~\cite{ntsb:AWE1549}.
%     \item Checklists may take too long to carry out - Swissair 111
% \end{itemize}

Testing procedures in checklists is often neglected by designers.~\cite{nasa:design}
This is shown in historic incidents, where the checklists to aid resolve the problem
at the time was not fit for the specific scenario that crew was in.

An example of this is the checklist used on US Airways Flight 1549. This flight
suffered a dual engine failure due to a bird strike at an altitude of \qty{2818}{ft} (\qty{859}{\meter}).
The first action by the pilot was to turn on the Auxiliary Power Unit (APU), allowing
critical systems, such as the flight controls and navigational aids, to be powered as
the engines were no longer able to power those systems. However, if the first call
was to run through the dual engine failure checklist (the one used on the flight),
it would have been the \nth{11} item on the checklist.
Using the checklist from the beginning could have resulted in a worse outcome of the incident,
but due to the crew's experience, they managed to execute the most successful
ditching (water landing) in history.~\cite{ntsb:AWE1549}

Therefore, this calls for a way to implement a way to test checklists for aspects
that may have been overlooked during the development of the checklist.

\section{Aim}
The goal of this project is to test checklists in Quick Reference Handbooks (QRH)
for flaws that could compromise the aircraft and making sure that the tests can
be completed in a reasonable amount of time by pilots. It is also crucial to make
sure that the tests are reproducible in the same flight conditions and a variety of
flight conditions.

\section{Objectives}
\begin{enumerate}
  \item Research current checklists that may be problematic and are testable
    in the QRH tester being made
  \item Implement a formal model that runs through checklists, with the
    research gathered, to produce an accurate test
    \begin{enumerate}
      \item Understand the relative states of the aircraft that need to be captured
      \item Ensure that the results of the checklist procedures are consistent
    \end{enumerate}
  \item Implement a QRH tester manager that
    \begin{enumerate}
      \item Runs the formal model and reacts to the output of the formal model
      \item Connect to a flight simulator to run actions from the formal model
      \item Implement checklist procedures to be tested, run them, and get
        feedback on how well the procedure ran
    \end{enumerate}
\end{enumerate}

\end{document}

