\documentclass[../dissertation.tex]{subfiles}

\begin{document}
\section{Solution Stack}
\begin{itemize}
  \item There would be around 3 main components to this tester
    \begin{itemize}
      \item Formal Model
      \item Flight Simulator plugin
      \item Checklist Tester (to connect the formal model and flight simulator)
    \end{itemize}
  \item As VDM-SL is being used, it uses VDMJ to parse the model~\cite{vdmj}. This was a starting
    point for the tech stack, as VDMJ is also open source.
  \item VDMJ uses Java, therefore my language of choice was a language related to Java.
\end{itemize}

\subsection{Formal Model}
\begin{itemize}
  \item There were a few ways of implementing the formal model into another application
  \item Some of these methods were provided by Overture~\cite{overture-remote}
    \begin{itemize}
      \item RemoteControl interface
      \item VDMTools API~\cite{vdmtoolbox-api}
    \end{itemize}
  \item However, both of these methods did not suit what was required as most of the
    documentation for RemoteControl was designed for the Overture Tool IDE. VDMTools
    may have handled the formal model differently
  \item The choice was to create a VDMJ wrapper, as the modules are available on Maven
\end{itemize}

\subsection{Checklist Tester}
\begin{itemize}
  \item VDMJ uses Java, meaning the logical choice would be to use something with Java
  \item As the tester is going to include a UI, the language choice was still important
  \item Kotlin~\cite{kotlin} was the choice in the end as Google has been putting Kotlin first
    compared to Java, it includes less boilerplate code (e.g. getters and setters)~\cite{android-kotlin}
  \item There are a variety of GUI libraries to consider using
    \begin{itemize}
      \item JavaFX~\cite{javafx}
      \item Swing~\cite{flatlaf}
      \item Compose Multiplatform~\cite{compose}
    \end{itemize}
  \item The decision was to use Compose Multiplatform in the end, due to time limitations and
    having prior experience in using Flutter~\cite{flutter}
  \item Compose Multiplatform has the ability to create a desktop application and a server,
    which would allow for leeway if a server would be needed
\end{itemize}

\subsection{Flight Simulator Plugin}
\begin{itemize}
  \item There are two main choices for flight simulators that can be used
    for professional simulation
    \begin{itemize}
      \item X-Plane~\cite{x-plane}
      \item Prepar3D~\cite{p3d}
    \end{itemize}
  \item X-Plane was the choice due to having better documentation for the SDK, and a variety
    of development libraries for the simulator itself
  \item For the plugin itself, there was already a solution developed by NASA, X-Plane Connect~\cite{xpc}
    that is more appropriate due to the time limitations and would be more likely to be reliable
    as it has been developed since 2015
\end{itemize}


\end{document}
