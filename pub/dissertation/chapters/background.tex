\documentclass[../dissertation.tex]{subfiles}

\begin{document}

%%%%% HYPOTHESIS %%%%%
\section{Hypothesis}
\begin{itemize}
    \item Checklists can be tested in a simulated environment 
      to find flaws in checklist for things like
      \begin{itemize}
        \item Can be done in an amount of time that will not endanger aircraft
        \item Provides reproducible results
        \item Procedures will not endanger aircraft or crew further (Crew referring to Checklist Manifesto with the cargo door blowout)
      \end{itemize}
    \item Results in being able to see where to improve checklists
\end{itemize}



%%%%% SAFETY %%%%%
\section{Safety in Aviation}
\subsection{History}
% TODO write about:
% - Safety became more of a concern when more passengers and more planes in the sky
% - Safety procedures being added
% - Rates of accidents
\begin{itemize}
  \item 70-80\% of aviation accidents are attributed to human factors~\cite{faa:reasons}
  \item The first use of a checklist was in 1935 after the crash of a prototype plane known
    back then as the Model 299 (known as the Boeing B-17 today), due to the complex procedures
    required to operate the aircraft normally and forgetting a step resulting in
    lack of controls during takeoff~\cite{manifesto}
  \item It was found that because of the complicated procedure to operate the aircraft
    that the pilots would forget steps, and hence the concept of checklists was tested,
    and found to minimize human errors~\cite{manifesto}
\end{itemize}

\subsection{Checklists}
% \begin{itemize}
%   \item Checklists are defined by the Civil Aviation Authority (CAA) as:
%     \blockquote{A set of written procedures/drills covering
%       the operation of the aircraft by the flight
%       crew in both normal and abnormal
%       conditions.~\ldots~The Checklist is
%       carried on the flight deck.}~\cite{caa:design}


%   \item Checklists have been shown to aid in minimizing human errors~\cite{manifesto}
  

%   \item However, according to the Civil Aviation Authority (CAA), the UK's aviation regulator:
%   \begin{itemize}
%     \item Checklists can be misleading and compromise the safety of the aircraft
%       due to them being either too confusing or taking too long to complete~\cite{nasa:design}
%     \item Other problems may include the crew skipping a step either unintentionally or by interruption,
%       or just failing to complete the checklist outright
%     \item The crew may also not be alerted to performance issues within the aircraft,
%       that running the checklist may cause~\cite{caa:design}
%   \end{itemize}
  
%   \item However, it is important to note that checklists does not prevent the human
%     factor of failure to use a checklist, like in the case of Northwest Airlines
%     Flight 255, where the National Transportation Safety Board (NTSB), an investigatory board
%     for aviation accidents in the United States, determined that
%     \enquote{the probable cause of the accident was the flight crew's failure
%     to use the taxi checklist to ensure that the flaps and slats were extended for takeoff.}~\cite{ntsb:NWA255}

%   \item These checklists can be bundled into a Quick Reference Handbook (QRH)
%     which the CAA defines it as:
%     \blockquote{A handbook containing procedures which
%       may need to be referred to quickly and/or
%       frequently, including Emergency and
%       Abnormal procedures. The procedures
%       may be abbreviated for ease of reference
%       (although they must reflect the procedures
%       contained in the AFM\footnote{
%         Aircraft Flight Manual - \enquote{The Aircraft Flight Manual produced by the
%         manufacturer and approved by the CAA.
%         This forms the basis for parts of the
%         Operations Manual and checklists. The
%         checklist procedures must reflect those
%         detailed in the AFM.}~\cite{caa:design}
%         }).
%       The QRH is often
%       used as an alternative name for the
%       Emergency and Abnormal Checklist.~\cite{caa:design}}

%   \item Therefore, as there may be a need for the checklist to be
%     referenced quickly and potentially in emergency situations,
%     these checklists should be tested for flaws 
% \end{itemize}

Checklists are defined by the Civil Aviation Authority (CAA),
the UK's aviation regulator, as:
%
\blockquote{A set of written procedures/drills covering
  the operation of the aircraft by the flight
  crew in both normal and abnormal
  conditions.~\ldots~The Checklist is
  carried on the flight deck.}~\cite{caa:design}
These checklists as a result has shown to be a crucial tool in aviation
to minimize human errors.~\cite{manifesto}

There are multiple checklists that are designed for aircraft for the use of
normal operation and potential problems that could arise during the flight.
These checklists are stored in a Quick Reference Handbook (QRH) which is
kept in the cockpit of each aircraft for use when needed. The definition
of a QRH by CAA is:
%
\blockquote{A handbook containing procedures which
  may need to be referred to quickly and/or
  frequently, including Emergency and
  Abnormal procedures. The procedures
  may be abbreviated for ease of reference
  (although they must reflect the procedures
  contained in the AFM\footnote{
    Aircraft Flight Manual - \enquote{The Aircraft Flight Manual produced by the
    manufacturer and approved by the CAA.
    This forms the basis for parts of the
    Operations Manual and checklists. The
    checklist procedures must reflect those
    detailed in the AFM.}~\cite{caa:design}
  }).
  The QRH is often
  used as an alternative name for the
  Emergency and Abnormal Checklist.~\cite{caa:design}}

However, checklists themselves can have design flaws as noted by researchers at
the National Aeronautics and Space Administration (NASA) where checklists
can be misleading, too confusing, or too long to complete, as a result
having the potential of compromising the safety of the aircraft.~\cite{nasa:design}
An example of this is what happened on Swiss Air Flight 111, where an electrical fault
was made worse by following the checklist, resulting in the aircraft crashing in the ocean.
This was as the flight crew was unaware of the severity of the fire caused by the
electrical fault. Following the steps in the checklist, one of the steps was
to cut out power to \enquote{non-essential} systems, which increased the
amount of smoke in the cockpit.
Simultaneously, the checklist itself was a distraction as it was found to take
around 30 minutes to complete in testing during the investigation.~\cite{tsb:SWR111}
This incident shows that checklists need to be tested for these flaws, and considering
the original checklist for Swiss Air Flight 111 would have taken 30 minutes
to theoretically complete, this could be time-consuming for checklist designers,
and this would be something to note whilst working on this project.

There are other potential problems with checklists,
noted by the CAA, where the person running through the checklist could skip a step
either unintentionally, by interruption, or just outright failing to complete the
checklist. Or the crew may also not be alerted to performance issues within the aircraft,
which would be a result of running the checklist.~\cite{caa:design} Therefore,
this would be useful to add for features when testing checklists, such as
adding the ability to intentionally skip a step of a checklist or gathering
statistics on how the performance of the aircraft has been affected as a result
of using the checklist.

Another problem to note about checklists is the human factor where the crew
may fail to use the checklist, like in the case of Northwest Airlines Flight 255,
where the National Transportation Safety Board (NTSB), an investigatory board
for aviation accidents in the United States, determined that
\enquote{the probable cause of the accident was the flight crew's failure
to use the taxi checklist to ensure that the flaps and slats were extended for takeoff.}~\cite{ntsb:NWA255}
This shows that even though checklists have shown to improve safety of the aircraft,
there are other measures that aviation regulatory bodies are required implement, to avoid
situations where the crew may completely ignore safety procedures and systems.


%%%%% FORMAL METHODS %%%%%
\section{Formal Methods}
% TODO add quick overview of what Formal Methods are
% Potential resource: https://shemesh.larc.nasa.gov/fm/fm-what.html

% \begin{itemize}
%   \item Formal methods is a mathematical technique that can be used towards the
%     verification of a system~\cite{nasa:formal}
%   \item This can be used to verify correctness of all the inputs in a system~\cite{nasa:formal}
%   \item Hence, as dealing with safety, it would be beneficial to have
%     the logic of this testing tool verified, to avoid bugs and misleading results
%   \item Airbus also uses formal methods in their avionics systems validation and
%     verification process~\cite{airbus:formal}
%   \item Some examples where Airbus used formal methods was during the development
%     for the Airbus A380, where they used it for proof of absence of stack overflows
%     and analysis of the numerical precision and stability of floating-point operators
%     to name a few~\cite{airbus:formal}
%   \item There are a multitude of specification languages, each of them
%     having their own reasons % TODO don't know if this should be included
% \end{itemize}

Formal methods is a mathematical technique that can be used towards the verification
of a system, that could either be a piece of software or hardware.
Therefore, this can be used to verify correctness of all the inputs in a system.~\cite{nasa:formal}
Hence, as this project is dealing with safety, it would be beneficial to use
formal methods for testing and verification.

An example of where formal methods is used within aviation is by Airbus, where
it was used during the development of the Airbus A380. Formal methods was used to test
the A380 for proof of absence of stack overflows and analysis of the numerical precision
and stability of floating-point operators to name a few.~\cite{airbus:formal}

% TODO maybe add a section about the amount of specification languages that exist?


%%%%% SOLUTION STACK %%%%%
\section{Solution Stack}
\begin{itemize}
  \item There would be around 3 main components to this tester
    \begin{itemize}
      \item Formal Model
      \item Flight Simulator plugin
      \item Checklist Tester (to connect the formal model and flight simulator)
    \end{itemize}
  \item As VDM-SL is being used, it uses VDMJ to parse the model~\cite{vdmj}. This was a starting
    point for the tech stack, as VDMJ is also open source.
  \item VDMJ is written in Java~\cite{vdmj}, therefore to simplify implementing VDMJ into the
    Checklist Tester, it would be logical to use a Java virtual machine (JVM) language.
\end{itemize}

\subsection{Formal Model}
\begin{itemize}
  \item There were a few ways of implementing the formal model into another application
  \item Some of these methods were provided by Overture~\cite{overture-remote}
    \begin{itemize}
      \item RemoteControl interface
      \item VDMTools API~\cite{vdmtoolbox-api}
    \end{itemize}
  \item However, both of these methods did not suit what was required as most of the
    documentation for RemoteControl was designed for the Overture Tool IDE. VDMTools
    may have handled the formal model differently
  \item The choice was to create a VDMJ wrapper, as the modules are available on Maven
\end{itemize}

\subsection{Checklist Tester}
\subsubsection{JVM Language}
\begin{itemize}
  \item There are multiple languages that are made for or support JVMs~\cite{jvm-alt-lang}
  \item Requirements for language
    \begin{itemize}
      \item Be able to interact with Java code because of VDMJ
      \item Have Graphical User Interface (GUI) libraries
      \item Have good support (the more popular, the more resources available)
    \end{itemize}
  \item The main contenders were Java and Kotlin~\cite{kotlin}
  \item Kotlin~\cite{kotlin} was the choice in the end as Google has been putting Kotlin first
    instead of Java. Kotlin also requires less boilerplate code (e.g. getters and setters)~\cite{android-kotlin}

\end{itemize}

\subsubsection{Graphical User Interface}
\begin{itemize}
  \item As the tester is going to include a UI, the language choice was still important
    \item There are a variety of GUI libraries to consider using
    \begin{itemize}
      \item JavaFX~\cite{javafx}
      \item Swing~\cite{flatlaf}
      \item Compose Multiplatform~\cite{compose}
    \end{itemize}
  \item The decision was to use Compose Multiplatform in the end, due to time limitations and
    having prior experience in using Flutter~\cite{flutter}
  \item Compose Multiplatform has the ability to create a desktop application and a server,
    which would allow for leeway if a server would be needed
\end{itemize}

\subsection{Flight Simulator Plugin}
\begin{itemize}
  \item There are two main choices for flight simulators that can be used
    for professional simulation
    \begin{itemize}
      \item X-Plane~\cite{x-plane}
      \item Prepar3D~\cite{p3d}
    \end{itemize}
  \item X-Plane was the choice due to having better documentation for the SDK, and a variety
    of development libraries for the simulator itself
  \item For the plugin itself, there was already a solution developed by NASA, X-Plane Connect~\cite{xpc}
    that is more appropriate due to the time limitations and would be more likely to be reliable
    as it has been developed since 2015
\end{itemize}


\end{document}
