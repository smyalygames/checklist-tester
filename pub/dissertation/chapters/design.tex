\documentclass[../dissertation.tex]{subfiles}

\begin{document}
\section{Abstraction}
\begin{figure}[!h]
\centering
  \begin{tikzpicture} [align=center, node distance=4cm]
    \node (connector) [box] {Checklist Tester Initerface};
    \node (server) [box, below of=connector] {Checklist Tester Server};
    \node (plugin) [box, right of=server] {Simulator Connector Plugin};
    \node (formal) [box, left of=server] {Formal Method};
    \node (simulator) [box, below of=plugin] {Flight Simulator};
  
    \draw [arrow] (server) -- (connector);
    \draw [arrow] (formal) -- (server);
    \draw [arrow] (plugin) -- (server);
    \draw [arrow] (plugin) -- (simulator);
  \end{tikzpicture}
  \caption{Abstract layout of components}
\end{figure}

\section{Model}
\begin{itemize}
  \item Formal modelling is the heart of the logic for testing checklists
  \item Formal model created using VDM-SL
  \item It allows to test that the checklists have been completed in a proper manner
    - and that it is provable
  \item Model keeps track of
    \begin{itemize}
      \item Aircraft state
      \item Checklist state
    \end{itemize}
  \item If an error were to occur in the model, this can be relayed that there was
    something wrong with running the test for the checklist, such as:
    \begin{itemize}
      \item Procedure compromises integrity of aircraft
      \item There is not enough time to complete the procedure
      \item There is a contradiction with the steps of the checklist
    \end{itemize}
\end{itemize}

\end{document}
