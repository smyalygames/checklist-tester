\documentclass[../dissertation.tex]{subfiles}

\begin{document}
\section{Abstraction}
\begin{figure}[!h]
\centering
  \begin{tikzpicture} [align=center, node distance=4cm]
    \node (connector) [box] {Checklist Tester};
    \node (plugin) [box, right of=connector] {X-Plane Connect};
    \node (formal) [box, left of=connector] {Formal Method};
    \node (simulator) [box, below=0.75cm of plugin] {X-Plane 12};
  
    \draw [arrow] (formal) -- (connector);
    \draw [arrow] (plugin) -- (connector);
    \draw [arrow] (plugin) -- (simulator);
  \end{tikzpicture}
  \caption{Abstract layout of components}
\end{figure}

\section{Model}
\begin{itemize}
  \item Formal modelling is the heart of the logic for testing checklists
  \item Formal model created using VDM-SL
  \item It allows to test that the checklists have been completed in a proper manner
    - and that it is provable
  \item Model keeps track of
    \begin{itemize}
      \item Aircraft state
      \item Checklist state
    \end{itemize}
  \item If an error were to occur in the model, this can be relayed that there was
    something wrong with running the test for the checklist, such as:
    \begin{itemize}
      \item Procedure compromises integrity of aircraft
      \item There is not enough time to complete the procedure
      \item There is a contradiction with the steps of the checklist
    \end{itemize}
\end{itemize}

\section{Scenarios}
\begin{itemize}
  \item Use a Quick Reference Handbook (QRH) to find potential list of checklists to test
  % TODO find these accident reports
  \item Look at previous accident reports that had an incident related to checklists
    and test it with my tool to see if it will pick it up
  \item These previous accident reports can be good metrics to know what statistics to
    look out for
\end{itemize}

\section{Decisions}
\end{document}
